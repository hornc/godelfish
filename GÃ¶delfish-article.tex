\documentclass{article}
\usepackage[utf8]{inputenc}
\usepackage{amsmath}% http://ctan.org/pkg/amsmath
\usepackage{amsfonts} 
\usepackage{amssymb}
\usepackage{seqsplit}

\begin{document}

\title{Gödelfish}
\author{https://esolangs.org/wiki/User:Salpynx}

\maketitle

\begin{abstract}
The abstract text goes here.
\end{abstract}

\section{Introduction}
Gödelfish is a Gödel numbering for the esoteric programming language Deadfish. It also provides a range of output encodings for the results of executing the code.

A Gödelfish program, $\ddot{\varphi}$, can be formed from a Deadfish program by making the following subtitutions and interpreting the result as a base4 numeral.

\begin{table}[h!]
  \begin{center}
    \caption{Gödel numbering.}
    \label{tab:table1}
    \begin{tabular}{c|c} % <-- Changed to S here.
      \textbf{Deadfish} & \textbf{Gödelfish (base4)} \\
      \hline
      d & 0\\
      i & 1\\
      s & 2\\
      o & 3\\
    \end{tabular}
  \end{center}
\end{table}

\section{Conversion to Brainfoctal}

A Gödelfish program, $\ddot{\varphi}$, can be converted to a corresponding Brainfoctal (a Turing complete Gödel numbering language) value, $\beta$, using the following bijective continuous piecewise linear function:

\[
b:\mathbb{R} \to \mathbb{N} \text{ given by }
b(\ddot{\varphi}) = \beta
\]

\[
b^{-1}:\mathbb{N} \to \mathbb{R} \text{ given by }
b^{-1}(\beta) = \ddot{\varphi}
\]

\begin{align}
	b(\ddot{\varphi}) ={}& \sum_{i=0}^{\lfloor \log_4 \ddot{\varphi} \rfloor + 1} b_{\textsc{idso}}(\lfloor{\frac{\ddot{\varphi}}{4^i}}\rfloor \bmod 4) \cdot 8^{\sum_{j=0}^{i} \lfloor \log_8(b_{\textsc{idso}}(\lfloor{\frac{\ddot{\varphi}}{4^j}}\rfloor \bmod 4)) \rfloor + 1} \\
	b_{\textsc{idso}}(x) ={}& \max(b_{\textsc{id}}(x), b_{\textsc{so}}(x)) \\
	b_{\textsc{id}}(x) ={}& 216x \cdot 8^{136} + d \\
	b_{\textsc{so}}(x) ={}& ox - g \\
	\begin{split}
	d ={}& \text{\footnotesize 32370779665404705561807609489961735142772450595278948496424386} \\
	     & \text{\footnotesize 66768114984669819898111048909997575431027847057652566964551746}
	\end{split} \\
	\begin{split}
	o ={}& \text{\footnotesize 14405520897770861239295320965768628031155278608853989995878087} \\
	     & \text{\footnotesize 51039303575288379014650106747233088457451983932729811175340555} \\
	     & \text{\footnotesize 027052154015192163827170902870107520892608}
	\end{split} \\
	\begin{split}
	g ={}& \text{\footnotesize 26754243754103765292330832085990828424744263601285353899531352} \\
	     & \text{\footnotesize 52873368656423787611508483038332610047342918664129411722483469} \\
	     & \text{\footnotesize 658221319869868861186264640741264481336638}
	\end{split}
\end{align}

This converted number can then be executed as Brainfoctal, and will produce the expected Deadfish output.

The crossover point for the two linear equations that make up $b_{\textsc{idso}}(x)$ is roughly 1.9

\begin{equation}
	b_{\textsc{idso}}(x) =
		\begin{cases}
			b_{\textsc{id}}(x),& \text{if } x \lessapprox 1.9 \\
			b_{\textsc{so}}(x),& \text{if } x \gtrapprox 1.9 \\
		\end{cases}
\end{equation}


\subsection{Variants}
There are two variants of Gödelfish.

$\ddot{\varphi} \in \mathbb{N}$: \textit{Natural} Gödelfish.

$\ddot{\varphi} \in \mathbb{R}$: \textit{Real} Gödelfish.

Where
$\ddot{\varphi} \in (\mathbb{R} - \mathbb{N})$ can be termed \textit{Unnatural} Gödelfish.


\section{Gödelfish Code Generation}
The following function generates a Gödelfish program that sets the accumulator to $i, \text{where } i \in \mathbb{N}_{i\ne256}$:

\begin{align}
	\phi(i) ={}& \frac{4^{\frac{\lvert\alpha(i)\rvert + \alpha(i)}{2}} - 1}{3}
	+ \gamma(i)(38^2 + 2)4^{\lvert i-17^2\rvert}
	+ \frac{0}{(4^4 - i)(\lvert i \rvert + i)}
	\intertext{where}
\begin{split}
	\alpha(i) ={}& i \big\lvert 1-\gamma(i) \big\rvert
	+ \gamma(i)(i - 17^{2})
\end{split}
	\intertext{and $\gamma: \mathbb{N} \to \{0, 1\}$ given by}
\begin{split}
	\gamma(i) ={}& \left\lceil\frac{i - 4^{4}}{1 + (i - 4^{4})^{2}}\right\rceil
\end{split}
\end{align}

This is by no means an optimised conversion, but it does produce accurate output for all valid inputs.
There are other possible code generation functions.


To modify a program to output the accumulator value, simply multiply by 4, and add 3:

\begin{equation}
	\ddot{\varphi}_{\textsc{output}} = 4\phi(i) + 3
\end{equation}


\section{Evaluation and output encoding}

\begin{align}
	O(\ddot{\varphi}, r, d) ={}& \left\lfloor \frac{E(\ddot{\varphi}, r^d)}{r^d} \right\rfloor
	\intertext{where}
\begin{split}
	E(\ddot{\varphi}, z) ={}& \sum_{i=0}^{\lfloor \log_4 \ddot{\varphi} \rfloor + 2} I(E(s(\ddot{\varphi}, i, 2), z), c(\ddot{\varphi}, i), z)
\end{split} \\
\begin{split}
	s(\ddot{\varphi}, i, n) ={}& \left\lfloor \frac{\ddot{\varphi}}{4^{\lfloor \log_4 \ddot{\varphi} \rfloor - i + n}} \right\rfloor
\end{split} \\
\begin{split}
	c(\ddot{\varphi}, i) ={}& s(\ddot{\varphi}, i, 1) - 4s(\ddot{\varphi}, i, 2)
\end{split} \\
\begin{split}
	I(s, c, z) ={}& v(s \bmod z, d(s, c, z))
\end{split} \\
\begin{split}
	v(a, x) ={}&
		\begin{cases}
			-a,& \text{if } a + x < 0\\
			-a,& \text{if } a + x = 256\\
			a + x
		\end{cases}
\end{split} \\
\begin{split}
	d(s, c, z) ={}&
		\begin{cases}
			-1,& \text{if } c = 0\\
			 1,& \text{if } c = 1\\
			(s \bmod z)^2 - s \bmod z,& \text{if } c = 2\\
			s(z - 1),& \text{if } c = 3\\
		\end{cases}
\end{split}
\end{align}

and
\begin{tabular}{rl}
	$\ddot{\varphi}$ =& Gödelfish value, $\ddot{\varphi} \in \mathbb{N}$ \\
	r =& Radix of output values. \\
	d =& Number of digits per output value in base radix.
\end{tabular}


\section{Conclusion}
Gödelfish is TG.

\end{document}
